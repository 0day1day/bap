\section{Overview}

The \bap infrastructure is implemented in C++ and OCaml.  The
front-end lifting is implemented primarily in C++.  The back-end is
implemented in OCaml. We interface the C++ front-end with the OCaml
back-end using OCaml via IDL generated stubs.

The front-end interfaces with Valgrind's
VEX~\cite{nethercote:2004:phd} to help lift instructions%% (see
%% Chapter~\ref{vine:frontend})
, GNU BFD for parsing executable objects,
and GNU libopcodes for pretty-printing the disassembly. Each
disassembled instruction is lifted to the IL, then passed to the
back-end .  

The \bap top-level directory structure is:
\begin{itemize}\squish
\item {\bf libasmir:} This directory contains all the code that
  interfaces which parses a binary file (when appropriate), interfaces
  with Valgrind's VEX~\cite{nethercote:2004:phd}, and produces an
  initial IL.  We are planning to migrate away from most of the
  C++ code in this directory to OCaml.

\item {\bf ocaml:} This directory contains the core \bap library and
  routines. All the code is written in ocaml. Only core \bap code
  should go on this directory. If you must modify it in a
  project-specific way, please make a project-specific branch for
  those changes.

\item {\bf utils:} This contains user utilities, again written in OCaml.

\item {\bf doc:} The directory containing this documentation, as well
  as ocamldoc generated documentation. (Note: ocamldoc needs to be run
  after compiling.)
\end{itemize}
