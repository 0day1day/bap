
%\renewcommand{\baselinestretch}{1.0}\normalsize

\begin{table}
{\bf Contexts}\\
\begin{tabular}{lllp{4in}}
  {\it Instruction} & $\Pi$ & $n$ $\mapsto$ \emphkind{instr} &
  Maps an instruction address to an instruction.\\
  {\it Variable} & $\Delta$ &  $id \mapsto \emphkind{var}$  &Maps
  a variable ID to its value.\\
  {\it Labels} & $\Lambda$ & $\emphkind{label\_kind} \mapsto n$ & Maps
  a label to the address of the corresponding {\tt label} instruction number.
\end{tabular}\\
\newline
{\bf Notation}\\
\begin{tabular}{lp{4.2in}}
  $\Delta \vdash e \Downarrow v$ & Expression $e$ evaluates to value
  $v$ given variable context $\Delta$ as given by the expression
  evaluation rules.\\
  $\Delta' = \Delta[x \leftarrow v]$ & $\Delta'$ is the same as
  $\Delta$ except extended to map $x$  to $v$.\\
  $\Pi \vdash p:i$ & $\Pi$ maps instruction address $p$ to instruction
  $i$. If $p \notin \Pi$, the machine gets stuck.\\
  $\Lambda \vdash v:p$ & $\Lambda$ maps instruction label $v$ to instruction
  address $p$. If $v \notin \Lambda$, then machine gets stuck. In
  addition, a well-formed machine should have $\Pi \vdash p : i$
  where $i = \texttt{label $v$}$, otherwise the machine is stuck.\\
%   $(\Pi, \Delta, p, \iota)$ & An abstract machine configuration.
%   $\Pi$ and $\Delta$ are contexts define above, $p$ is the program
%   counter, and $\iota$ is the current instruction.\\  
  $(\Delta, p, i) \leadsto (\Delta', p', i')$ &  An
  execution step. $p$ and $p'$ are
  the pre and post step program counters, $i$ and $i'$ are the pre and
  post step  instructions, and $\Delta$ and $\Delta'$ are the pre and post
  step variable contexts. Note $\Lambda$ and $\Pi$ are currently static, thus
  for brevity not included in the execution context.\\
\end{tabular}
\caption{Operational semantics notation.}
\end{table}
\begin{table}
{\bf Instructions}
\begin{small}
\[
\begin{array}{cc}
  \infer[\textsc{Assign}]
    { % conclusion
      \Delta, p, \text{x := $e$} 
      \leadsto 
      \Delta', p+1, i
    }
    { % premise
      \Delta \vdash e \Downarrow v 
      & \Delta' = \Delta[x \leftarrow v]
      & \Pi \vdash p+1:i
    } &
  \infer[\textsc{Jmp}]
  { % conclusion
    \Delta, p, \texttt{jmp($\ell$)} 
    \leadsto 
    \Delta, p', \texttt{label $v$}
  }
  { % premise
    \Delta \vdash \ell \Downarrow v
    & \Lambda \vdash v : p'
    & \Pi \vdash p' : \texttt{label $v$}
  }\\[10pt]
  \infer[\textsc{label}]
  { % conclusion
    \Delta, p, \texttt{label $\ell$} 
    \leadsto
    \Delta, p+1, i
  }
  { % premise
      \Pi[p+1] = i
  } &
  \infer[\textsc{halt}]
  { % conclusion
    \Delta, p, \texttt{halt $e$} 
    \leadsto
    \text{terminate with  $v$}
%    \Pi, \Delta, p+1, \texttt{halt $v$}
  }
  {
     \Delta \vdash e \Downarrow v
  }\\[10pt]
  \infer[\textsc{assert-t}]
  { % conclusion
    \Delta, p, \texttt{assert}(e)
    \leadsto
    \Delta, p+1, i
  } 
  { % premise
    \Delta \vdash e \Downarrow 1 &
    \Pi[p+1] = i
  } &
  \infer[\textsc{assert-f}]
  { % conclusion
    \Delta, p, \texttt{assert}(e)
    \leadsto
    \text{terminate with  $\perp$}
  }
  { % premise
    \Delta \vdash e \Downarrow 0
  }
\\
\end{array}
\]
\[
\begin{array}{c}
  \infer[\textsc{CJmp-t}]
  { %conclusion
    \Delta, p, \texttt{cjmp($e$, $\ell_T$, $\ell_F$)} 
    \leadsto
    \Delta, p',  \texttt{label $v$}
  }
  { %premise
    \Delta \vdash e \Downarrow 1
    & \Delta \vdash \ell_T \Downarrow v
    & \Lambda \vdash v: p'
    & \Pi \vdash p' :  \texttt{label $v$}
  }\\[10pt]
  \infer[\textsc{CJmp-f}]
  { %conclusion
    \Delta, p, \texttt{cjmp($e$, $\ell_T$, $\ell_F$)} 
    \leadsto
    \Delta, p', \texttt{label $v$}
  }
  { %premise
    \Delta \vdash e \Downarrow 0
    & \Delta \vdash \ell_F \Downarrow v
    & \Lambda \vdash v: p'
    & \Pi \vdash p' : \texttt{label $v$}
  } \\[10pt]
  \text{No rule for {\tt special}, when
    $v \notin \Lambda$, and when  $p \notin \Pi$.}  
\end{array}
\]
\end{small}

{\bf Expressions}\\
\begin{small}
\[
\begin{array}{cc}
  \infer[\textsc{binop}]
  { % conclusion
    \Delta \vdash e_1 \Diamond_b e_2 \Downarrow v
  }
  { % premise
    \Delta \vdash e_1 \Downarrow v_1 
    & \Delta \vdash e_2 \Downarrow v_2
    & v = v_1 \Diamond_b v_2
  } &
  \infer[\textsc{unop}]
  { % conclusion
    \Delta \vdash \Diamond_u e_1 \Downarrow v
  } 
  { % premise
    \Delta \vdash e_1 \Downarrow v_1 
    & v = \Diamond_u v_1
  }\\[10pt]
\end{array}
\]
\[
\begin{array}{ccc}
  \infer[\textsc{Value}]
  { % conclusion
    \Delta \vdash v \Downarrow v
  } 
  { % premise (empty)
  } &
  \infer[\textsc{Var}]
  { % conclusion
    \Delta \vdash x \Downarrow v
  }
  { % premise
    \Delta \vdash x:v
  } &
  \infer[\textsc{let}]
  { % conclusion
    \Delta \vdash \texttt{let $x$ = $e_1$ in $e_2$} \Downarrow v
  }
  { % premise
    \Delta \vdash e_1 \Downarrow v_1 
    & \Delta' = \Delta[x \leftarrow v_1]
    & \Delta' \vdash e_2 \Downarrow v
  }\\
\end{array}
\]
\[
\begin{array}{c}
  \infer[\textsc{Load$_{\text{little}}$}]
  { % conclusion
    \Delta \vdash \texttt{load($e_1$, $e_2$, $e_3$, $\tau_{\text{reg}}$)} \Downarrow v
  }
  { % premise
    \Delta \vdash e_1 \Downarrow v_1 
    & \Delta \vdash e_2 \Downarrow v_2
    & \Delta \vdash e_3 \Downarrow 0
    & n = \text{\# bytes of $\tau_{\text{reg}}$}
    & v = v_1[v_2..v_2+n] \text{ in little endian byte order}
  } \\[10pt]
  \infer[\textsc{Load$_{\text{big}}$}]
  { % conclusion
    \Delta \vdash \texttt{load($e_1$, $e_2$, $e_3$, $\tau_{\text{reg}}$)} \Downarrow v
  }
  { % premise
    \Delta \vdash e_1 \Downarrow v_1 
    & \Delta \vdash e_2 \Downarrow v_2
    & \Delta \vdash e_3 \Downarrow 1
    & n = \text{\# bytes of $\tau_{\text{reg}}$}
    & v = v_1[v_2..v_2+n] \text{ in big endian byte order}
  } \\[10pt]
  \infer[\textsc{Load$_\text{array}$}]
  { % conclusion
    \Delta \vdash \texttt{load($e_1$, $e_2$, $e_3$, \texttt{array\_t})} \Downarrow v
  }
  { % premise
    \Delta \vdash e_1 \Downarrow v_1 
    & \Delta \vdash e_2 \Downarrow v_2
    & \Delta \vdash e_3 \Downarrow 0
    & v = v_1[v_2]
  } \\[10pt]
  \infer[\textsc{Store$_{\text{little}}$}]
  { % conclusion
    \Delta \vdash \texttt{store($e_1$, $e_2$, $e_3$, $e_4$, $\tau_{\text{reg}}$)}  \Downarrow v
  }
  { % premise
    \Delta \vdash e_1 \Downarrow v_1 
    & \Delta \vdash e_2 \Downarrow v_2
    & \Delta \vdash e_3 \Downarrow v_3
    & \Delta \vdash e_4 \Downarrow 0
    & n = \text{\# bytes $\tau_{\text{reg}}$}
    & v = v_1[v_2..v_2+n \leftarrow v_3] \text{ (little endian)}
  }\\[10pt]
  \infer[\textsc{Store$_{\text{big}}$}]
  { % conclusion
    \Delta \vdash \texttt{store($e_1$, $e_2$, $e_3$, $e_4$, $\tau_{\text{reg}}$)}  \Downarrow v
  }
  { % premise
    \Delta \vdash e_1 \Downarrow v_1 
    & \Delta \vdash e_2 \Downarrow v_2
    & \Delta \vdash e_3 \Downarrow v_3
    & \Delta \vdash e_4 \Downarrow 1
    & n = \text{\# bytes $\tau_{\text{reg}}$}
    & v = v_1[v_2..v_2+n \leftarrow v_3] \text{ (big endian)}
  }\\[10pt]
  \infer[\textsc{Store$_{\text{array}}$}]
  { % conclusion
    \Delta \vdash \texttt{store($e_1$, $e_2$, $e_3$, $e_4$, \texttt{array\_t})}  \Downarrow v
  }
  { % premise
    \Delta \vdash e_1 \Downarrow v_1 
    & (e_1 : \texttt{array\_t})
    & \Delta \vdash e_2 \Downarrow v_2
    & \Delta \vdash e_3 \Downarrow v_3
    & (e_4 \text{ ignored})
    & v = v_1[v_2 \leftarrow v_3]
  }
\end{array}\\
\]
\[
\begin{array}{cc}
  \infer[\textsc{cast$_u$}]{ %conclusion
    \Delta \vdash \texttt{cast(unsigned, $\tau_{\text{reg}}$, $e$)}
    \Downarrow v
  }
  { %premise
   \Delta \vdash e \Downarrow v & \text{zero extend $v$ to
    $\tau_{\text{reg}}$ bits}
  } &
  \infer[\textsc{cast$_s$}]{ %conclusion
    \Delta \vdash \texttt{cast(signed, $\tau_{\text{reg}}$, $e$)}
    \Downarrow v
  }
  { %premise
   \Delta \vdash e \Downarrow v & \text{sign extend $v$ to
    $\tau_{\text{reg}}$ bits}
  }\\[10pt]
    \infer[\textsc{cast$_h$}]{ %conclusion
    \Delta \vdash \texttt{cast(high, $\tau_{\text{reg}}$, $e$)}
    \Downarrow v
  }
  { %premise
   \Delta \vdash e \Downarrow v & \text{extract 
    $\tau_{\text{reg}}$  high bits of $v$}
  } &
    \infer[\textsc{cast$_l$}]{ %conclusion
    \Delta \vdash \texttt{cast(low, $\tau_{\text{reg}}$, $e$)}
    \Downarrow v
  }
  { %premise
   \Delta \vdash e \Downarrow v & \text{extract 
    $\tau_{\text{reg}}$  low bits of $v$}
  }\\[10pt]
  \infer[\textsc{name}]
  { % conclusion
    \cdot \vdash \texttt{name}\text{(string)} : \text{string}
  }
  { % premise
  } &
  \infer[\textsc{unknown}]
  { % conclusion
    \cdot \vdash \texttt{unknown}(s) \Downarrow \perp
  }
  { % premise
  }
\end{array}
\]
\end{small}
\caption{Operational Semantics.}
\label{vine:taboperational}
\end{table}

%\renewcommand{\baselinestretch}{1.66}\normalsize

%%% Local Variables: 
%%% mode: latex
%%% TeX-master: "../main"
%%% End: 
