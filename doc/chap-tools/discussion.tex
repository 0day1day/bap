\section{Discussion}
\label{vine:discussion}

\subsection{Why Design a New Infrastructure?}  

At a high level, we designed \bap as a new platform because existing
platforms are a) defined for higher-level languages and thus not
appropriate for binary code analysis, b) designed for orthogonal goals
such as binary instrumentation or decompilation, and/or c) unavailable
to use for research purposes.  As a result, other tools we tried
(e.g., DynInst~\cite{dyninst} version 5.2, Phoenix~\cite{phoenix}
April 2008 SDK, IDA Pro~\cite{idapro} version 5, and others) were
inadequate, e.g., could not create a correct control flow graph for
the 3 line assembly shown in Figure~\ref{vine:shl-add}. Another reason
we created \bap is so that we could be sure of the semantics of 
analysis. Existing platforms tend not to have publicly available
formally defined semantics, thus we would not know exactly what we are
using.

Formal semantics are important for several practical reasons. We found
that defining the semantics of \bap was helpful in catching design
bugs, unsupported assumptions, and other errors that could affect many
different kinds of analysis. In addition, the semantics are helpful
for communicating how \bap works with other researchers.  Further,
without a specified semantics, it is difficult to show an analysis is
correct. For example, in ~\cite{brumley:2006:alias} we show our
proposed assembly-level alias analysis~\cite{brumley:2006:alias} is
correct with respect to the operational semantics of \bap.


\subsection{Limitations of \bap}

\begin{description}
\item[Architectures] \bap only supports x86 and x86-64 right now.  We intend to
  add support for ARM in the future.
\item[Semantics] \bap is designed to enable program analysis of binary
  program states. Therefore, analyses that depend upon more than the
  operational semantics of the instruction, such as timing or non-deterministic behavior, fall outside the scope of
  \bap. 
\item[User-mode instructions] \bap only models user-mode level
  instructions.  Unhandled instructions will be lifted as {\tt
    special} statements or {\tt unknown} expressions.
\item[Integer instructions] \bap only models instructions that
  manipulate integers.  In particular, \bap does not model floating
  point instructions.
\end{description}

\subsection{Is Lifting Correct?}

Assembly instructions are lifted to \bap in a syntax directed manner.
One may view the \bap IL as a model of the underlying assembly.  There
is a chance, however, that lifting could produce incorrect
IL. Although it is impossible to say that all assembly instructions
are correctly lifted, the advantage of \bap's design is that only the
lifting process needs to understand the semantics of the original
assembly.

We perform nightly testing to make sure that \bap's model of execution
matches what happens on a real x86 processor.  Our nightly tests also
tell us which instructions are used in our test programs that are not
modeled in \bap.  The results of these tests are always available at
\url{http://bap.ece.cmu.edu/nightly-reports/}.

\subsection{Size of Lifting IL for a Program}

A single assembly instruction will typically be translated into
several \bap statements. Thus, the resulting \bap program will have
more statements than the corresponding assembly.  For example and
roughly speaking, x86 assembly instructions are raised to be about 7
\bap statements: 1 \bap statement for the direct effect, and 6 for
updating processor-specific status flags.

In our experience, the constant-size factor in code size is worth the
benefits of simplifying the semantics of assembly.  Assembly
instructions are designed to be efficient for a computer to execute,
not for a human to understand. The IL, on the other hand, is designed
to be easy for a human to understand.  We have found even experienced
assembly-level programmers will comment that they have a hard time
keeping track of control and data dependencies since there are few
syntactic cues in assembly to help. \bap, on the other hand, obviates
all data and control dependencies within the code. 

%%% Local Variables: 
%%% mode: latex
%%% TeX-master: "../main"
%%% End: 
