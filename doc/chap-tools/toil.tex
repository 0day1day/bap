\section{\texttt{toil}: Lifting Binary Code to \bil}

The {\tt toil} tool lifts binary code to \bil.  Lifting up binary code
to the IL consists of two steps:
\begin{itemize}
\item First, the binary file is opened using libbfd. BAP can read
  binaries from any bfd-supported binary, including ELF and PE
  binaries.

\item Second, the executable code in the binary is disassembled.  {\tt
  toil} currently uses a linear sweep disassembler.

\item Each assembly instruction discovered by the linear sweep disassembler
  is then lifted directly to \bil.
\end{itemize}

Lifted assembly instructions have all of the side-effects explicitly
exposed.  As a result, a single typical assembly instruction will be
lifted as a sequence of \bil instructions.  For example, the {\tt add
  \$2, \%eax} instruction is lifted as:

\begin{centering}
\begin{scriptsize}
\begin{verbatim}
addr 0x0 @asm "add    $0x2,%eax"
label pc_0x0
T_t1:u32 = R_EAX:u32
T_t2:u32 = 2:u32
R_EAX:u32 = R_EAX:u32 + T_t2:u32
R_CF:bool = R_EAX:u32 < T_t1:u32
R_OF:bool = high:bool((T_t1:u32 ^ ~T_t2:u32) & (T_t1:u32 ^ R_EAX:u32))
R_AF:bool = 0x10:u32 == (0x10:u32 & (R_EAX:u32 ^ T_t1:u32 ^ T_t2:u32))
R_PF:bool =
  ~low:bool(R_EAX:u32 >> 7:u32 ^ R_EAX:u32 >> 6:u32 ^ R_EAX:u32 >> 5:u32 ^
            R_EAX:u32 >> 4:u32 ^ R_EAX:u32 >> 3:u32 ^ R_EAX:u32 >> 2:u32 ^
            R_EAX:u32 >> 1:u32 ^ R_EAX:u32)
R_SF:bool = high:bool(R_EAX:u32)
R_ZF:bool = 0:u32 == R_EAX:u32
\end{verbatim}
\end{scriptsize}
\end{centering}

The lifted \bil code explicitly detail all the side-effects of the
{\tt add} instruction, including all six flags that are updated by the
operation.  As another example, an instruction with the {\tt rep}
prefix (whose semantics are in Figure~\ref{vine:rep}) is lifted as a
sequence of instructions that form a loop.

In addition to binary files, {\tt toil} can also lift an instruction
trace to the IL.  The most recent BAP trace format can be lifted using
the {\tt -serializedtrace} option.

{\tt toil} can output to several formats for easy parsing, including
protobuf, JSON, and XML.  These formats are selected via the {\tt
  -topb}, {\tt -tojson}, and {\tt -toxml} options.  Code for reading
the protobuf encoding is in the {\tt piqi-files/protobuf} directory.

%%% Local Variables: 
%%% mode: latex
%%% TeX-master: "../main"
%%% End: 
