\section{\texttt{toil}: Lifting Binary Code to \bil}

The {\tt toil} tool lifts binary code to \bil.  Lifting up binary code
to the IL consists of two steps:
\begin{itemize}\squish
\item First, the binary file is opened using libbfd. BAP can read binaries from any bfd-supported binary, including ELF and PE binaries.  

\item Second, the executable code in the binary is disassembled.  {\tt
  toil} currently uses a linear sweep disassembler.

\item Each assembly instruction discovered by the linear sweep disassembler
  is then lifted directly to \bil.
\end{itemize}

Lifted assembly instructions have all of the side-effects explicitly
exposed.  As a result, a single typical assembly instruction will be
lifted as a sequence of \bil instructions.  For example, the {\tt add
  eax,0x2} x86 instruction is lifted as:

\begin{centering}
\begin{scriptsize}
\begin{tightcode}
tmp1 = EAX;
EAX = EAX + 2;
//eflags calculation
CF = (EAX<tmp1);
tmp2 = cast(low, EAX, reg8\_t);
PF = (!cast(low,
              ((((tmp2>>7)^(tmp2>>6))^((tmp2>>5)^(tmp2>>4)))^
              (((tmp2>>3)^(tmp2>>2))^((tmp2>>1)^tmp2)))), reg1\_t);
AF = (1==(16\&(EAX^(tmp1^2))));
ZF = (EAX==0);
SF = (1==(1\&(EAX>>31)));
OF = (1==(1\&(((tmp1^(2^0xFFFFFFFF))\&(tmp1^EAX))>>31)));
\end{tightcode}
\end{scriptsize}
\end{centering}

The lifted \bil code explicitly detail all the side-effects of the
{\tt add} instruction, including all 6 {\tt eflags} that are updated
by the operation.  As another example, an instruction with the {\tt
  rep} prefix (whose semantics are in Figure~\ref{vine:rep}) is lifted
as a sequence of instructions that form a loop.

In addition to binary files, {\tt toil} can also lift an instruction trace
to the IL.

%%% Local Variables: 
%%% mode: latex
%%% TeX-master: "../main"
%%% End: 
