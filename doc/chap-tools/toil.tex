\section{Lifting Binary Code to \bil}

The {\tt toil} tool lifts binary code to \bil.  The syntax for {\tt
  toil} is yet to be determined.


\section{The internals of {\tt toil}}
Lifting up binary code to the IL consists of three steps:
\begin{itemize}\squish
\item {\bf Step 1.} First, the binary file is disassembled. {\tt toil}
  currently interfaces with %% two disassemblers: IDA Pro~\cite{idapro},
  %% a commercial disassembler, and 
  a linear sweep disassembler built on top of GNU libopcodes.
  Interfacing with other disassemblers like IDA Pro is straight-forward.

\item {\bf Step 2.} The disassembly is passed to VEX, a third-party
  library which turns assembly instructions into the VEX intermediate
  language. The VEX IL is part of the Valgrind dynamic instrumentation
  tool~\cite{nethercote:2004:phd}.  The VEX IL is also similar to a
  RISC-based language. As a result, the lifted IL has only a few
  instruction types, similar to \bap. However, the VEX IL  is
  insufficient for performing program analysis because it lacks
  information about the implicit side effects of instructions, e.g., what
  {\tt eflags} are set by x86 instructions.  This step is mainly
  performed in order to simplify the development of \bap: we let the
  existing tool take care of reducing assembly instructions to a
  basic IL, then step 3 exposes all implicit side-effects so that
  the analysis is faithful.

  

\item {\bf Step 3.} We lift the VEX IL to \bil.  The resulting \bil
  IL is intended to be faithful to the semantics of the disassembled
  assembly instructions. 
\end{itemize}

Lifted assembly instructions have all of the side-effects explicitly
exposed.  As a result, a single typical assembly instruction will be
lifted as a sequence of \bil instructions.  For example, the {\tt add
  eax,0x2} x86 instruction is lifted as:

\begin{centering}
\begin{scriptsize}
\begin{tightcode}
tmp1 = EAX;
EAX = EAX + 2;
//eflags calculation
CF:reg1\_t = (EAX<tmp1);
tmp2 = cast(low, EAX, reg8\_t);
PF = (!cast(low,
              ((((tmp2>>7)^(tmp2>>6))^((tmp2>>5)^(tmp2>>4)))^
              (((tmp2>>3)^(tmp2>>2))^((tmp2>>1)^tmp2)))), reg1\_t);
AF = (1==(16\&(EAX^(tmp1^2))));
ZF = (EAX==0);
SF = (1==(1\&(EAX>>31)));
OF = (1==(1\&(((tmp1^(2^0xFFFFFFFF))\&(tmp1^EAX))>>31)));
\end{tightcode}
\end{scriptsize}
\end{centering}

The lifted \bil code explicitly detail all the side-effects of the
{\tt add} instruction, including all 6 {\tt eflags} that are updated
by the operation.  As another example, an instruction with the {\tt
  rep} prefix (whose semantics are in Figure~\ref{vine:rep}) is lifted
as a sequence of instructions that form a loop.

In addition to binary files, {\tt toil} can also lift an instruction trace
to the IL. Conditional branches in a trace are lifted as {\tt assert}
statements so that the executed branch is followed. This is done to
prevent branching outside the trace to an unknown instruction. The
TEMU~\cite{temu} dynamic analysis tool currently supports generating
traces in the \bap trace format.



