\section{Functions}

In this tutorial, we show how to use the get\_functions command to
extract functions from a binary with debugging symbols.  Specifically,
we will look at the \cmdline{/bin/ls} command.  First, make sure you
have a version of the binary with symbols; the version that comes with
your distribution probably had its symbols stripped.  If we know the
name of a function we want to extract, such as
\cmdline{hash\_table\_ok}, we can extract it using
\cmdline{get\_functions -unroll 3 /bin/ls hash\_table\_ok}.  This will
produce a file called \texttt{resolvehash\_table\_ok.il}. We used the
\cmdline{-unroll} option to remove loops by unrolling; this is
important when generating VCs, because generating VCs is generally
only possible for acyclic programs. If we wanted to see if the
function can return zero, we could use topredicate:
\cmdline{topredicate -q -il resolvehash\_table\_ok.il -post 'R\_EAX\_32:u32 ==
  0:u32' -stp-out /tmp/f -solve}.

%%% Local Variables: 
%%% mode: latex
%%% TeX-master: "../main"
%%% End: 
