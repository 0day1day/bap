\section{Traces}

In this section, we will explain how to use our PIN-based trace
recording tool to record a trace.  We then show how to perform common
trace analysis tasks.

\subsection{Setup}

The PIN trace tool is located in the \cmdline{pintraces/} directory.
Because of licensing reasons, we cannot distribute PIN with the trace
tool.  The PIN tool cannot be build until PIN has been extracted
inside of the \bap directory: if \bap is located in
\cmdline{\$PATH/bap}, then PIN should be in
\cmdline{\$PATH/bap/pin}.  On Linux, running the ./getpin.sh script
from the \cmdline{\$PATH/bap/pintraces} directory will
automatically download and extract PIN for Linux; the user is
responsible for accepting the PIN license agreements.

On Windows, the process is more complicated.  We usually test with
Windows 7, but some people have reported success with Windows XP SP3
as well.  First, make sure that \cmdline{\$PATH} contains no
spaces. Then, install GNU Make for Windows
(\url{http://gnuwin32.sourceforge.net/packages/make.htm}) and Visual
C++ 2010 Express
(\url{http://www.microsoft.com/visualstudio/en-us/products/2010-editions/visual-cpp-express}).
Make sure to add the directory containing \texttt{make} (the default
is \verb!C:\Program Files\GnuWin32\bin!) to the Windows \texttt{Path}
environment variable.  Also download PIN and extract it to the
\cmdline{\$PATH/bap/pin} directory.  Next, we need to upgrade the
Visual Studio project for Google's protobuffers library, which our PIN
tool depends on\footnote{This can be done automatically at the command
  line using devenv.exe, but this tool is only part of Visual Studio
  proper; it does not come with the Express versions.}.  To do this,
navigate to the
\cmdline{\$PATH/bap/libtracewrap/libtrace/protobuf/vsprojects}
directory and open \cmdline{protobuf.sln}.  When the Visual Studio
Conversion Wizard appears, click Finish and close the summary window.
It is normal for the conversion process to have warnings, but not
errors.  Next, right click on the libprotobuf project in the Solution
Explorer, and select Properties.  Select the Release configuration at
the top of the dialog box, and then navigate to the C/C++ Code
Generation settings.  Change the Runtime Library to Multi-threaded
(/MT)\footnote{If you skip this step, you will get unresolved symbol
  errors while linking the PIN tool}.  Finally, close Visual Studio
and save the changes to the project.  Open a Visual Studio command
prompt, navigate to \cmdline{\$PATH/bap/libtracewrap/libtrace/src/cpp}
and run \cmdline{make -f Makefile.windows} to build protobuffers and
the libtrace library.

The PIN tool itself can be built by executing \cmdline{make} in the
\cmdline{\$PATH/bap/pintraces} directory on both Windows and Linux.
After compilation, the PIN tool should exist in
\\ \cmdline{\$PATH/bap/pintraces/obj-ia32/gentrace.so} (or
gentrace.dll on Windows).  In the rest of the chapter, we will assume
Linux is being used; most interaction with the trace tool is the same.

\subsection{Recording a trace}

To see the command line options to the trace tool, execute 

\begin{verbatim} 
$PATH/bap/pin/pin -t \
 $PATH/bap/pintraces/obj-ia32/gentrace.so -help -- /bin/ls
\end{verbatim}

By default, the trace tool will only log instructions that are
\emph{tainted}, i.e., those that depend on user input.  The options
that begin with taint are used to mark various data
as being user input.  For instance, -taint-files readme marks the file
readme as being user input.

We will record a trace of a simple buffer overflow.  Run 

\begin{verbatim}
echo ``helloooooooooooooooooo'' > readme
\end{verbatim}

 to create the input file.  Then run 

\begin{verbatim}
make $PATH/bap/pintraces/examples/bof1/
\end{verbatim}

and

\begin{verbatim}
$PATH/bap/pin/pin -t  \
 $PATH/bap/pintraces/obj-ia32/gentrace.so -taint-files readme \
 -- $PATH/bap/pintraces/examples/bof1/bof1
\end{verbatim}

The PIN tool will output
many debugging messages; this is normal.  If the trace tool detected
the buffer overflow, it will print ``stack smashing detected'' near
the end of the logs.  At this point, there should be a trace file
ending with suffix bpt in the current working directory.  In the
following commands, we assume this file is named trace.bpt.

To lift the trace data and print it, run 

\begin{verbatim}
iltrans -serializedtrace trace.bpt -pp-ast /dev/stdout
\end{verbatim}

It is also possible to concretize the trace, which removes jumps and performs
memory concretization, by executing 

\begin{verbatim}
iltrans -serializedtrace trace.bpt -trace-concrete -pp-ast /dev/stdout
\end{verbatim}

Adding the -trace-check option before -trace-concrete causes BAP to compare its
internal evaluator's notion of state with the actual values recorded in the
trace.  It can be used to check for bugs in the IL.  Finally, running 

\begin{verbatim}
iltrans -serializedtrace trace.bpt -trace-formula f
\end{verbatim}

will symbolically execute the trace and output the generated verification
condtion to the file f. This can then be solved with stp to find satisfying
answers to the trace.
