\section{Traces}

In this section, we will explain how to use our PIN-based trace
recording tool to record a trace.  We then show how to perform common
trace analysis tasks.

\subsection{Setup}

The PIN trace tool is located in the \cmdline{pintraces/} directory.
Because of licensing reasons, we cannot distribute PIN with the trace
tool.  The PIN tool cannot be built until PIN has been extracted
inside of the \bap directory.  In the following, we assume that
\cmdline{\$BAPDIR} is set to the \bap directory. For example, if you
extracted \bap to \cmdline{/home/user/Downloads/bap-x.y}, then you should
replace \cmdline{\$BAPDIR} below with
\texttt{/home/user/Downloads/bap-x.y}. Once downloaded, PIN should be
extracted to \cmdline{\$BAPDIR/pin}.  On Linux, running the
./getpin.sh script from the \cmdline{\$BAPDIR/pintraces}
directory will automatically download and extract PIN for Linux; the
user is responsible for accepting the PIN license agreements.

On Windows, the process is more complicated.  We usually test with
Windows 7 and Windows XP SP3, but expect NT, 2000, 2003, Vista, and 7
to work.  First, make sure that \cmdline{\$BAPDIR} contains no
spaces. Then, install GNU Make for Windows
(\url{http://gnuwin32.sourceforge.net/packages/make.htm}) and Visual
C++ 2010 Express
(\url{http://www.microsoft.com/visualstudio/en-us/products/2010-editions/visual-cpp-express}).
Make sure to add the directory containing \texttt{make} (the default
is \verb!C:\Program Files\GnuWin32\bin!) to the Windows \texttt{Path}
environment variable.  Also download PIN and extract it to the
\cmdline{\$BAPDIR/pin} directory.  Next, we need to upgrade the Visual
Studio project for Google's protobuffers library, which our PIN tool
depends on\footnote{This can be done automatically at the command line
  using devenv.exe, but this tool is only part of Visual Studio
  proper; it does not come with the Express versions.}.  To do this,
navigate to the
\cmdline{\$BAPDIR/libtracewrap/libtrace/protobuf/vsprojects} directory
and open \cmdline{protobuf.sln}.  When the Visual Studio Conversion
Wizard appears, click Finish and close the summary window.  It is
normal for the conversion process to have warnings, but not errors.
Next, right click on the libprotobuf project in the Solution Explorer,
and select Properties.  Select the Release configuration at the top of
the dialog box, and then navigate to the C/C++ Code Generation
settings.  Change the Runtime Library to Multi-threaded
(/MT)\footnote{If you skip this step, you will get unresolved symbol
  errors while linking the PIN tool}.  Finally, close Visual Studio
and save the changes to the project.  Open a Visual Studio command
prompt, navigate to \cmdline{\$BAPDIR/libtracewrap/libtrace/src/cpp}
and run \cmdline{make -f Makefile.windows} to build protobuffers and
the libtrace library.

The PIN tool itself can be built by executing \cmdline{make} in the
\cmdline{\$BAPDIR/pintraces} directory on both Windows and Linux.
After compilation, the PIN tool for recording x86 programs should exist in \\
\cmdline{\$BAPDIR/pintraces/obj-ia32/gentrace.so} (or
\cmdline{gentrace.dll} on Windows).  If you are running on a x86-64
host, you should also have a PIN tool for recording x86-64 programs in\\
\cmdline{\$BAPDIR/pintraces/obj-intel64/gentrace.so}.  In the rest of
the chapter, we will assume Linux is being used; most interaction with
the trace tool is the same on Windows.

\subsection{Recording a trace}

To see the command line options to the trace tool, execute 

\begin{verbatim}
$BAPDIR/utils/gentrace.bash -help -- /bin/ls
\end{verbatim}

By default, the trace tool will only log instructions that are
\emph{tainted}, i.e., those that depend on user input.  The
\cmdline{-taint-*} command line options are used to mark various
inputs as being tainted.  For instance, \cmdline{-taint-files readme}
marks the file \cmdline{readme} as being tainted.

We will record a trace of a simple buffer overflow.  Run 

\begin{verbatim}
echo "helloooooooooooooooooooooooooooooooooo" > readme
\end{verbatim}

 to create the input file.  Then run

\begin{verbatim}
$BAPDIR/utils/gentrace.bash -taint-files readme \
 -- $BAPDIR/tests/C/bof1 readme
\end{verbatim}

The PIN tool will output
many debugging messages; this is normal.  If the trace tool detected
the buffer overflow, it will print ``Stack smashing detected'' near
the end of the logs.  At this point, there should be a trace file
ending with suffix bpt in the current working directory.  In the
following commands, we assume this file is named trace.bpt.

To lift the trace data and print it using \cmdline{iltrans} run:

\begin{verbatim}
iltrans -trace trace.bpt -pp-ast /dev/stdout
\end{verbatim}

\cmdline{iltrans} processes traces by lifting the entire trace into
memory, and then performing the specified operation(s).
Unfortunately, this makes it unsuitable for processing traces that do
not fit in memory.  \cmdline{streamtrans} is an alternative utility
for processing traces in a \emph{streaming} model.  Instead of lifting
the entire trace at once, \cmdline{streamtrans} loads only a small
part of it at a time.  The downside to \cmdline{streamtrans} is that
many traces capabilities in BAP are implemented in \cmdline{iltrans}
but not \cmdline{streamtrans}.

To lift the trace data and print it using \cmdline{streamtrans} use:

\begin{verbatim}
streamtrans -tracestream trace.bpt -pp-ast /dev/stdout
\end{verbatim}

Regardless of the lifting tool, the lifted trace data should look
something like:

\begin{verbatim}
addr 0x8048639 @asm "mov    0x804a038,%eax" @tid "0"
  @context "R_EAX_32" = 0x68, 1, u32, wr
  @context "mem32[0x804a038]" = 0x5, 0, u8, rd
  @context "mem32[0x804a039]" = 0x0, 0, u8, rd
  @context "mem32[0x804a03a]" = 0x0, 0, u8, rd
  @context "mem32[0x804a03b]" = 0x0, 0, u8, rd
label pc_0x8048639
R_EAX_32:u32 = mem32:?u32[0x804a038:u32, e_little]:u32
\end{verbatim}

In addition to the lifted IL, there are several trace-specific
annotations.  \texttt{@tid "0"} indicates that thread 0 executed this
instruction.  The \texttt{@context} attribute provides information
about this instruction's operands.  The four values in the tuple refer
to the operand's initial value, its taint identifier, data type, and
access type, respectively.  Positive taint identifiers indicate the
operand was only tainted by the taint source with the corresponding
number.  For example, taint identifier five generally corresponds to
the fifth symbolic byte introduced into the program.  An operand may
have a taint identifier of negative one if it is tainted by multiple
source bytes.  A taint identifier of zero corresponds to an untainted
operand.  The access type is one of read only (rd), write only (wr),
or both (rw).

It is also possible to concretize the trace after lifting, which
removes jumps and performs memory concretization, by executing

\begin{verbatim}
iltrans -trace trace.bpt -trace-concrete -pp-ast /dev/stdout
\end{verbatim}
or
\begin{verbatim}
streamtrans -tracestream trace.bpt -trace-concrete -pp-ast /dev/stdout
\end{verbatim}

Adding the -trace-check option before -trace-concrete causes BAP to
compare its internal evaluator's notion of state with the actual
values recorded in the trace.  It can be used to check for bugs in the
IL.  For example:

\begin{verbatim}
iltrans -trace trace.bpt -trace-check -trace-concrete
\end{verbatim}
or
\begin{verbatim}
streamtrans -tracestream trace.bpt -trace-check -trace-concrete-drop
\end{verbatim}


Finally, running
%
\begin{verbatim}
iltrans -trace trace.bpt -trace-formula f
\end{verbatim}
or
\begin{verbatim}
streamtrans -tracestream trace.bpt -trace-formula f
\end{verbatim}

will symbolically execute the trace and output the generated verification
condition to the file f. This can then be solved with stp to find satisfying
answers to the trace.

%%% Local Variables: 
%%% mode: latex
%%% TeX-master: "../main"
%%% End: 
